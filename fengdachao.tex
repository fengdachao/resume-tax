%-------------------------
% CV for Northeastern University Faculty in Latex
% Author : Zoe Kearney
% Based off of: https://www.overleaf.com/latex/templates/coles-resume-template/qhpynjcvjpcj
% Follows: https://theprofessorisin.com/2016/08/19/dr-karens-rules-of-the-academic-cv/
% License : MIT
%------------------------

% Document class and font size
\documentclass[a4paper,9pt]{extarticle}

% Packages
\usepackage[utf8]{inputenc} % For input encoding
\usepackage{geometry} % For page margins
\geometry{a4paper, margin=1in} % Set paper size and margins
\usepackage{titlesec} % For section title formatting
\usepackage{enumitem} % For itemized list formatting
\usepackage{hyperref} % For hyperlinks

\usepackage[UTF8]{ctex}

% Formatting
\setlist{noitemsep} % Removes item separation
\titleformat{\section}{\large\bfseries}{\thesection}{1em}{}[\titlerule] % Section title format
\titlespacing*{\section}{0pt}{\baselineskip}{\baselineskip} % Section title spacing

% Begin document
\begin{document}

% Disable page numbers
\pagestyle{empty}

% Header
\begin{center}
冯大超\\[3pt] % Name
\textbf{software engineer}\\[1pt] % CV
北京 | \href{mailto:fengdachao@aliyun.com}{fengdachao@aliyun.com} | 13311071079 % Contact info
\end{center}

%------------------------

% Education Section
\section*{教育经历}
\noindent
\newline
\textbf{本科} \\
2003 - 2007 \\ 
黑龙江大学-软件学院 \\

%------------------------

% Employment Section
\section*{工作经历}
\noindent
\newline
\textbf{高级前端工程师} \\
2021-2024 \\
北京万桥达观(Everbridge) \\
前端需求开发,主要以ant-design为前端组件库,react16.x为UI framework,基于webpack module federation的微前端架构。\\

\textbf{前端leader} \\
2020-2021 \\
智联招聘 \\ 
前端组内日常管理工作, 评审及进度跟踪。 维护zhaopin.com pc/m端日常维护, 需求迭代。 根据需求确定技术方案, 制定重构计划。 需求开发工作。 \\ 

\noindent
\textbf{高级前端工程师} \\
2017-2020 \\
Synchronoss Technologies \\ 
前端产品架构调整、升级 、和React架构性能分析和优化。 邮件客户端产品新需求实现包括邮件列表、日历、联系人模块, 基于React, Redux框架, TDD开发。 运用Code-spliting如Lazy load等技术优化性能。 使用Redux开发业务逻辑, Redux-saga处理Side-effect业务, 结合reducer完成需求。

\noindent
\textbf{前端工程师} \\
2016-2017  \\
未云科技 \\ 
Spa系统模块开发, 实现用户管理和用户数据管理功能等需求。使用AngularJS库完成产品开发, 调用后台微服务, 实现数据持久化 \\

%------------------------

% End document
\end{document}
