% Target Typeset Program = upLatex
% Encoding = UTF-8
% Lang = ja
% 2022年度版wordファイルはこちら https://docs.google.com/document/d/1N98Rauh7ueYK5xjZio0-yDCfGTi0hgTE/edit 
\documentclass[10pt,a4j,uplatex,twoside]{jsarticle}
\usepackage{zasshikai}

\begin{document}

% 1段組みにする場合,以下の "\twocolumn[" と,"]" をコメントアウトする.
\twocolumn[
    \begin{framed}
        {\fontsize{12pt}{3pt}\selectfont 
            \kintou{4.75zw}{発表者}:黒髪 二郎 (c9999@st.cs.kumamoto-u.ac.jp)\\
            \kintou{4.75zw}{所属}:夏目研究室\\
            \kintou{4.75zw}{論文題目}:Information for Paper Presentations in Zasshi-kai of CSEE\\
            \kintou{4.75zw}{著者}:Terute Rubose, Nanashino Gombey, and Heno Henomohedge\\
            \kintou{4.75zw}{掲載誌}:Transaction on Zasshikai, Vol.13, No.9, pp.855–860, 2022}
    \end{framed}
    \vspace{2em}
]

% 本文
\mysection{論文紹介要領}
    紹介する論文は,学会誌掲載もしくはこれに準ずる形で公開された論文で,指導教員の承認を得た論文とする. なお,複数の論文を紹介しても構わない.
 
    発表形式は,1人で発表する形式Aを基本とする(形式A:発表8分,質疑応答4分,計12分/1人).指導教員の判断により2人で発表する形式Bとすることもできる (形式B:発表16分,質疑応答8分,計24分/2人).
 
    発表,発表用スライドおよび論文概要に使用する言語は,日本語または英語のいずれかとする.ただし,発表で使用していない方の言語で質問されたことを理由に回答を拒否してはならない.

    発表時間には発表者交代の時間も含まれる.発表時間を超過する場合は,発表の途中であっても発表を止めて質疑応答を行う.

    司会は発表者の指導教員が行い,質疑応答は1つの論文の紹介が終わるごとに行う.尚,質問が出なかった場合は指導教員が行う.

\mysection{論文紹介前の準備}
    発表会場での回覧用のため,紹介する論文のコピーを1部,指導教員に渡しておく.
    また,紹介論文の論文概要を次の\ref{sakusei}節に従って作成する.
    その後,論文概要と発表資料を提出フォームより発表日全日までに登録する. 登録がない場合には発表を認めない.

\mysection{論文概要の作成方法}\label{sakusei}
    紹介する論文の概要を日本語または英語のいずれかで A4 判 2ページにまとめ,A4 判1枚に両面印刷して配布できるようにすること.

    また,論文をよく理解して概要を作成すること.原著論文の一部をそのまま丸写しされたものが概要として記載されるとは考え難く,諸君の良識に期待する.

    なお,この雑誌会実施要領は,諸君が作成する論文概要の様式に従って書かれている.以下の注意事項をよく読んで,論文概要を作成すること.
    \begin{itemize}
        \item 上下左右の余白を 20 mm ~ 25 mm 取ること.
        \item 1ページ目の上部に,この雑誌会要領に示すように,発表者の氏名 (メールアドレス) ,所属研究室,論文題目 (原語のみ) ,著者名,掲載された学会誌等の名称,巻数,号数,ページ番号 (最初と最後) ,掲載年を記載し,これらの項目を枠で囲むこと.
        \item その下に 2 行空白をおいて,論文概要の本文を書くこと.論文概要は,自分の言葉で分かりやすくまとめ,説明のために必要であれば,図面や式を用いてもよい.また,二段組みで記載してもよい.ただし,前項の論文題目や掲載誌など枠に囲った部分を二段組みにしてはならない.
        \item フォントは 10 pt 以上を使用する.この雑誌会要領は,1枚目の上部は 12 pt,本文は 10 pt のフォントで書かれている.%手書きする場合は,10 pt 相当以上の大きさで丁寧に記載すること.
        \item 引用の形式についても十分情報を含むように引用する \cite{howtocite}.図や表への参照も必ずつけること.図\ref{logo}は熊本大学ロゴへの参照を示す.
    \end{itemize}

\begin{figure}[htbp]
    \centering
    \includegraphics[width=1.0\linewidth]{logo.eps} 
    \caption{A Logo of Kumamoto University}
    \label{logo}
\end{figure}

\mysection{雑誌会当日の注意事項}
    雑誌会は,4年次必修科目「プレゼンテーション技術」として実施される.雑誌会に無断欠席した場合,以後の卒業研究の受講を認めない.

% 参考文献
% Style Option = {jplain, jalpha, jabbrv, junsrt, jipsj.bst, jorsj.bst}
\bibliographystyle{junsrt}
\bibliography{ref}
\end{document}
