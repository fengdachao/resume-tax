% Document class and font size
\documentclass[a4paper,9pt]{extarticle}

% Packages
\usepackage[utf8]{inputenc} % For input encoding
\usepackage{geometry} % For page margins
\geometry{a4paper, margin=1in} % Set paper size and margins
\usepackage{titlesec} % For section title formatting
\usepackage{enumitem} % For itemized list formatting
\usepackage{hyperref} % For hyperlinks
\usepackage[UTF8]{ctex} % support Chinese

% Formatting
\setlist{noitemsep} % Removes item separation
\titleformat{\section}{\large\bfseries}{\thesection}{1em}{}[\titlerule] % Section title format
\titlespacing*{\section}{0pt}{\baselineskip}{\baselineskip} % Section title spacing

% Begin document
\begin{document}

% Disable page numbers
\pagestyle{empty}

% Header
\begin{center}
冯大超\\[3pt] % Name
\textbf{前端/全栈工程师}\\[1pt] % CV
北京 | \href{mailto:fengdachao@aliyun.com}{fengdachao@aliyun.com} | 13311071079 % Contact info
\end{center}

\section*{个人简介}
\noindent
\newline
十年以上软件开发经验, 包括半年海外工作经历, 对Javascript语言及React框架有深入的理解, 熟悉BS, Restful业务模式, 开发领域涉及前端 的web开发(react,Vue,angular)以及后端的服务编写c sharp, node.js, Java, 可以独立设计搭建Web应用程序框架, 具有全栈的开发能力, 具有敏捷开发经验。 开发过的产品包括邮件web客户端, 工程自动化管理系统, 云平台管理web端, 保险出单系统 web端, 有团队合作精神和较强学习能力。\\

%------------------------

% Education Section
\section*{教育经历}
\noindent
\newline
\textbf{本科} \\
2003-09 - 2007-07 \\ 
黑龙江大学-软件学院 \\

%------------------------

% Employment Section
\section*{工作经历}
\noindent
\newline
\textbf{高级前端工程师} \\
北京万桥达观(Everbridge) 2021.07-2024.06 \\
Saas平台公司, 技术栈react antd webpack,微前端框架, aws平台管理, terraform \\
\textbf{职责} \\
前端需求开发,主要以ant-design为前端组件库,react16.x为UI framework,基于webpack module federation的微前端架构。 \\
\textbf{项目} \\
基于webpack module federation微服务框架,以react及typescript为基础语言实现saas平台管理系统功能页面的开发,得力于微服务框架实现各组独立开发,单独部署上线。\\

\noindent
\textbf{前端leader} \\
智联招聘 2020.06-2021.07 \\
互联网公司,技术栈VUE, 内部大前端编译组件, 上线平台 \\ 
\textbf{职责} \\
前端组内日常管理工作, 评审及进度跟踪。 维护zhaopin.com pc/m端日常维护, 需求迭代。 根据需求确定技术方案, 制定重构计划。 需求开发工作。 \\ 
\textbf{项目} \\
对To C端zhaopin.com web/h5端日常维护、更新新需求迭代.\\

\noindent
\textbf{高级前端工程师} \\
Synchronoss Technologies 2017.06-2020.01 \\
纳斯达克上市公司, 软件产品包括邮件客户端, 公有云产品 \\
\textbf{职责} \\
前端产品架构调整、升级 、和React架构性能分析和优化。 邮件客户端产品新需求实现包括邮件列表、日历、联系人模块, 基于React, Redux框架, TDD开发。 运用Code-spliting如Lazy load等技术优化性能。 使用Redux开发业务逻辑, Redux-saga处理Side-effect业务, 结合reducer完成需求。\\
\textbf{项目} \\
以React, redux为框架对web邮件客户端进行的日常开发及维护工作, 2019.07 relocate印度工作到20年一月。\\
短暂参与Angular框架的Saas管理系统的新需求开发。\\

\noindent
\textbf{前端工程师} \\
未云科技 2016.03-2017.07  \\
云管理平台创业公司, 技术栈AngularJS, gulp, webpack\\
\textbf{职责} \\
Spa系统模块开发, 实现用户管理和用户数据管理功能等需求。使用AngularJS库完成产品开发, 调用后台微服务, 实现数据持久化 \\
\textbf{项目} \\
参与初创公司的基于AngularJS的云平台管理系统的前端开发工作。\\

\noindent
\textbf{软件工程师} \\
天波广电科技 2010.01-2016.03 \\
设备自动化管理产品公司, 主要技术栈Asp.net, Jquery, python, nodejs, angularjs \\
\textbf{职责} \\
Asp.net系统开发, python以太网数据收集服务, 参与与硬件系统调试, 及系统使用培训 实现人员权限, 报表模块, 使用Nodejs express框架, 实现部分公司业务, 如用户调查等。 产品升级包括后期维护工作。\\
\textbf{项目} \\
参与基于B/S机构的数字化监控管理平台的设计开发,主要是用ASP.net,和Sql server数据库,前端页面是actions script构建的flash动画页面。\\
以python语言为基础的网络数据解析服务的开发维护工作。\\

%------------------------

% End document
\end{document}
